\documentclass[a4paper, 12pt]{article}
\usepackage[a4paper,left=2cm,right=2cm,top=3cm,bottom=3cm]{geometry}
\usepackage[ngerman]{babel}
\usepackage{graphicx}
\usepackage{amsmath}
\usepackage{tikz}
\usepackage{amsmath}
\title{Spidercam}
\date{April 2020}
\author{Andreas Stumpf, Sönke Bartelo}
\begin{document}
	\maketitle
	\newpage
	\tableofcontents
	\pagebreak
	\listoffigures
	\pagebreak
	\section{Einleitung}
	Um automatisierte Kameraführungen in großen und schwer zugänglichen Orten, wie z.B. Stadien, zu ermöglichen werden Kameras an Seilen befestigt.
	In diesem Dokument soll eine solche Kamerasteuerung geplant und durchgeführt werden.
	Dabei wird eine theoretische Herleitung der Ortskoordinaten einer solchen Konstruktion durchgeführt und anschließend an einem Model verifiziert.
	Ziel ist es entweder durch Vorgabe eines Endpunktes eine automatisierte Bewegungssteuerung zum definierten Punkt, oder eine Verfolgung einer Solltrajektorie zu erreichen.
	Test für Git
	\pagebreak
	\section{Herleitung der inversen Kinematik Andi}
		\subsection{Betrachtung im Zweidimensionalen}
			\subsubsection{Definition der Punkte}
			Um das Anfangsproblem zu vereinfachen wird die Herleitung der Längenänderung des Seils eines Motors zunächst im zweidimensionalen durchgeführt. Das Übersetzen der Funktionen in den dreidimensionalen Raum ist damit trivial.
			Um eine Durchführbarkeit innerhalb der geometrischen Begrenzungen zu garantieren, wird ein Koordinatensystem so in den Raum gelegt, sodass sich für zwei der drei Eckpunkte ein x- Wert von Null ergibt. Damit lassen sich Begrenzungsfunktionen zwischen den Eckpunkten aufstellen, die eine einfache Berechnung der Machbarkeit zulassen. Der theoretische Aufbau der Konstruktion sieht dann wie folgt aus:
			\begin{figure}[h!]		
				\begin{tikzpicture}
					\coordinate 	[label=left:$0$]	(Nullpunkt) at (0,0);
					\coordinate		[label=below:$A$] 	(A) at (0,0);
					\coordinate		[label=below:$B$] 	(B) at (14.5,0);
					\coordinate		[label=right:$C$] 	(C) at (7.25,12);
					\coordinate		[label=right:$P$] 	(P) at (5,1.5);
					\coordinate		[label=right:$P`$] 	(P`) at (9,5.7);
					\coordinate		[label=below:$x$]	(X) at (15,0);
					\coordinate		[label=left:$y$]	(Y) at (0,12.5);
					\coordinate		[label=left:C_{y}]	(YC)at (0,12);
					\coordinate		[label=below:C_{x}]	(XC)at (7.25,0);
					\coordinate		[label=left:P_{y}]	(YP)at (0,1.5);
					\coordinate		[label=below:P_{x}]	(XP)at (5,0);
					\coordinate		[label=left:P^{`}_{y}]	(YP`)at (0,5.7);
					\coordinate		[label=below:P^{`}_{x}]	(XP`)at (9,0);
					\draw [->]	(Nullpunkt) -- (X);
					\draw [->]	(Nullpunkt) -- (Y);
					\draw 	(A) -- (B);
					\draw 	(B) -- (C) node[midway,right] {y$_{2}};
					\draw 	(C) -- (A) node[midway,left] {y$_{1}};
					\draw 	(A) -- (P) node[midway,above] {$\vec{a}$};
					\draw 	(A) -- (P`)node[midway,above] {$\vec{a^{´}}$};
					\draw 	(B) -- (P) node[midway,above] {$\vec{b}$};
					\draw 	(B) -- (P`)node[midway,above] {$\vec{b^{´}}$};
					\draw 	(C) -- (P) node[midway,left]  {$\vec{c}$};
					\draw 	(C) -- (P`)node[midway,left]  {$\vec{c^{´}}$};
					\draw 	(P) -- (P`)node[midway,left]  {$\vec{t}$};
					\draw [dashed]  (YP) -- (P);
					\draw [dashed]  (XP) -- (P);
					\draw [dashed]  (YP`) -- (P`);
					\draw [dashed]  (XP`) -- (P`);
					\draw [dashed]  (YC) -- (C);
					\draw [dashed]  (XC) -- (C);
					\fill (A) circle (2pt);
					\fill (B) circle (2pt);
					\fill (C) circle (2pt);
				\end{tikzpicture}
				\caption{Darstellung der Transformation eines Punktes}
			\end{figure}
			\pagebreak
			\subsubsection{Raumbegrenzung}
			Bevor eine Längenänderung ermittelt werden kann, muss zunächst garantiert werden, dass alle Punkte der Bewegung erreicht werden können. Da zuvor zwei der drei Eckpunkte auf die x- Achse gelegt wurden kann nun einfach ermittelt werden, ob alle Punkte unterhalb der Verbindungsgeraden 
			$\overline {AC} $ 
			und 
			$\overline {CB} $
			liegen.
			Dafür werden folgende Geradengleichungen aufgestellt:
			\begin{align}
				y $_{1} = m $_{1} * x + b $_{1}\\
				y $_{2} = m $_{2} * x + b $_{2}
			\end{align}
			Mit b$_{1}=0 und m$_{1}=$\frac{\Delta y}{\Delta x}$
			\begin{align}
				y $_{1} = \frac{c$_{y}-a$_{y}}{c$_{x}-a$_{x}}*x
			\end{align}
			Mit m$_{2}=$\frac{\Delta y}{\Delta x}$ ergibt sich y$_{2} zu:
			\begin{align}
				y $_{2} = \frac{c$_{y}-b$_{y}}{c_{x}-b_{x}}*x+b_{2}
			\end{align}
			Da die Koordinaten von Eckpunkt B bekannt sind, kann b_{2} mit Hilfe von m_{2} ermittelt werden.
			\begin{align}
				m_{2} = \frac{y_{2}-y_{1}}{x_{2}-x_{1}}
			\end{align}
			Mit b_{2}=y_{2},x_{2}=0 und m_{2}=$\frac{c_{y}-b_{y}}{c_{x}-b_{x}}$ folgt:
			\begin{align}
				b_{2} = \frac{(c_{y}-c_{x})*(c_{y}-b_{y})}{c_{x}-b_{x}}
			\end{align}
			Da nun beide Geradengleichungen definiert sind, kann überprüft werden, ob alle Punkte der Bewegung folgende Bedingungen erfüllen:
			\begin{align}
						p_{x}<c_{x} 
				\land 	p_{y}<y_{1}(p_{x}) 
				\lor
						p_{x} \geq c_{x} 
				\land 	p_{y}<y_{2}(p_{x})
			\end{align}
			\subsubsection{Berechnung der Längenänderungen}
			Ist garantiert, dass alle Punkte der Bewegung realisierbar sind kann die Berechnung der Seillängenänderung erfolgen. 
			Die Längenänderung der Steuerseile kann geometrisch bestimmt werden.
			Dabei wird davon ausgegangen, dass die aktuelle Kameraposition P, sowie die Montagepositionen der Aktuatoren A, B und C bekannt sind.\\
			Die Länge des Seils a zu Motor A kann bestimmt werden als Subtraktion des Vektors zu Motor A und des Vektors zu P.\\
			Die Länge des Seils a´ zu Motor A kann bestimmt werden als Subtraktion des Vektors zu Motor A und des Vektors zu P.\\
			Damit ergibt sich die Längenänderung des Seils zu Motor A als Differenz der Längen a und a´.
			Analog dazu ergeben sich die Längenänderungen der Seile zu Motor B und C.
			Soll ein Punkt mit drei Aktuatoren in der Ebene bewegt werden,
			so ergibt sich die Längenänderung $\Delta n$ in Abhängigkeit der Montageposition des Aktuators und der angestrebten Kameraposition zu:
			\begin{align}
				\Delta n =
				|\vec{n^{´}}|-|\vec{n}|=
				\sqrt{
				(P^{`}_{x}-N_{x})^2+
				(P^{`}_{y}-N_{y})^2
				}
			\end{align}
		\subsection{Betrachtung im Dreidimensionalen}
			\subsubsection{Definition der Punkte}
			Die Positionen der Motoren bleiben nach der Betrachtung im Zweidimensionalen identisch, jedoch wird jeder Motorposition eine Z- Koordinate hinzugefügt, sodass sich die Eckpunkte nun im Format N=(N_{x} N_{y} N_{z}) befinden. Die x- und y- Komponenten bleiben bestehen, um sämtliche Vereinfachungen aus dem Zweidimensionalen übernehmen zu können. 
			\subsubsection{Raumbegrenzung}
			Die Berechnung der räumlichen Begrenzung einer Bewegung im dreidimensionalen Raum kann äquivalent zur Bewegung im Zweidimensionalen durchgeführt werden. Die Begrenzung wird jedoch um folgende Bedingung erweitert:
			\begin{align}
				P_{z}<A_{z}\land P_{z}>0
			\end{align}
			Damit muss nun jeder Punkt einer Bewegung folgende Bedingungen erfüllen:
			\begin{align}
						p_{x}<c_{x} 
				\land 	p_{y}<y_{1}(p_{x})
				\land	P_{z}<A_{z}
				\land 	P_{z}>0 
				\lor
						p_{x} \geq c_{x} 
				\land 	p_{y}<y_{2}(p_{x})
				\land	P_{z}<A_{z}
				\land 	P_{z}>0
			\end{align}
			\subsubsection{Berechnung der Längenänderungen}
			Die Berechnung der Längenänderung ergibt sich analog zum Zweidimensionalen, wird jedoch um die Z- Komponente erweitert.
			\begin{align}
				\Delta n =
				|\vec{n^{´}}|-|\vec{n}|=
				\sqrt{
				(P^{`}_{x}-N_{x})^2+
				(P^{`}_{y}-N_{y})^2+
				(P^{`}_{z}-N_{z})^2
				}
			\end{align}
			\pagebreak	
	\section{Herleitung der inversen Kinematik }
		\subsection{Mechanische Konstruktion}
	Vereinfacht wird im folgenden Angenommen, dass sich die gesamte Anordnung der Kameraposition zunächst durch ein einfaches, gleichseitiges Dreieck annehmen lässt.
	Die Entscheidung zu Gunsten eines Dreiecks basiert auf der Tatsache, dass zur Befestigung einer Plattform im dreidimensionalen Raum mindesten drei Befestigungspunkte benötigt werden.
	Daraus resultiert die Wahl von drei Seilen zur Steuerung.
	
	\begin{figure}
		\centering
		\includegraphics[width=0.8\textwidth,draft]{mech_constr.png}
		\label{key}
		\caption{Die Mechanische Konstruktion mit Vereinfachungen.}
	\end{figure}

		\subsection{Inverse Kinematik}
		Zunächst soll an dieser Stelle die Pose der Kamera zu bestimmen. 
		Diese besteht aus den kartesischen Koordinaten
	
		\begin{align}
			\vec{p} = 
			\begin{pmatrix}
				p_x \\
				p_y \\
				p_z
			\end{pmatrix} , 
		\end{align}
	
		mit den jeweiligen Koordinaten in x,y und z Richtung.
		Die Parametrierung der Winkel erfolgt über Kardan-Winkel (Roll, Pitch, Yaw).
		Diese zeichnet sich dadurch aus, dass die Drehung um x-, y- und z-Achse jeweils um die alten Achsen durchgeführt wird.
		Daraus ergibt sich Bestimmung der Lage
	
		\begin{align}
			\vec{w}=
			\begin{pmatrix}
				\phi_e \\
				\theta_e \\
				\psi_e
			\end{pmatrix},
		\end{align}

		jeweils bezogen auf den Endeffektor-Frame.
	
		
\end{document}